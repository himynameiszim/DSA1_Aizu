\documentclass{report}

\usepackage{graphicx}
\usepackage{algorithm}
\usepackage{array}
\usepackage{dsfont}
\usepackage{algpseudocode}
\usepackage{listings}
\usepackage{amsmath}
\usepackage{tikz}
\usepackage{pdfpages}
\usepackage{float}

\DeclareMathOperator{\rank}{rank}
\makeatletter
\newenvironment{sqcases}{%
  \matrix@check\sqcases\env@sqcases
}{%
  \endarray\right.%
}
\def\env@sqcases{%
  \let\@ifnextchar\new@ifnextchar
  \left\lbrack
  \def\arraystretch{1.2}%
  \array{@{}l@{\quad}l@{}}%
}
\makeatother

\usetikzlibrary{calc}


\input{/mnt/fa80f336-3342-4d78-8bfd-a43e434a2cda/Latex/preamble.tex}
\input{/mnt/fa80f336-3342-4d78-8bfd-a43e434a2cda/Latex/macros.tex}
\input{/mnt/fa80f336-3342-4d78-8bfd-a43e434a2cda/Latex/letterfonts.tex}

\title{\Huge{FU08 \-- Automata and Languages}\\Exercise 2}
\author{\huge{NGUYEN Tuan Dung}\\\huge{s1312004}}
\date{December 8, 2024}

\begin{document}

\maketitle

% Cau 1

\qs{Answer the following question}{
For the set $S=\{1,4,7,8\}$ and the relations:
    $$
    \begin{aligned}
    & \mathrm{R} 1=\{(1,7),(1,8),(7,4),(4,7)\} \\
    & \mathrm{R} 2=\{(1,7),(7,1),(7,4),(4,7),(4,4)\} \\
    & \mathrm{R} 3=\{(1,1),(1,7),(4,4),(7,1),(7,4),(4,7),(7,7),(8,8)\} \\
    & \mathrm{R} 4=\{(1,1),(4,4),(7,7),(8,8)\}
    \end{aligned}
    $$
a. Complete the following table (by writing yes or no):\\
    \begin{tabular}{|l|l|l|l|l|}
    \hline & Reflexive & Symmetric & Transitive & Equivalence \\
    \hline R1 & & & & \\
    \hline R2 & & & & \\
    \hline R3 & & & & \\
    \hline R4 & & & & \\
    \hline
    \end{tabular}\\
\newline
b. Find the following:\\
    i. the reflexive closure for R1 and R2\\
    ii. the transitive closure for R2 and R4\\
    iii. the symmetric closure for R2 and R3\\
    iv. the \{reflexive, symmetric, transitive \}\--closure for R1, and R3.}

\sol{\newline
a.\\
  \begin{tabular}{|l|l|l|l|l|}
    \hline & Reflexive & Symmetric & Transitive & Equivalence \\
    \hline R1 &no &no &no &no \\
    \hline R2 &no &yes &no &no \\
    \hline R3 &yes &yes &no &no \\
    \hline R4 &yes &yes &yes &yes \\
    \hline
    \end{tabular}\\
\newline
b.\\
i. The reflexive closure for R1 is $\mathrm{R}1^{*} = \{(1,7);(1,8);(7,4);(4,7);(1,1);(4,4);(7,7);(8,8)\}$\\
The reflexive closure for R2 is $\mathrm{R}2^{*} = \{(1,7);(7,1);(7,4);(4,7);(4,4);(1,1);(7,7);(8,8)\}$\\
ii. The transitive closure for R2 is $\mathrm{R}2^{+} = \{(1,7);(7,1);(7,4);(4,7);(4,4);(1,1);(1,4);(7,7);(4,1)\}$\\
The transitive closure for R4 is $\mathrm{R}4^{+} = \{(1,1);(4,4);(7,7);(8,8)\}$\\
iii. The symmetric closure for R2 is $\mathrm{R}2_{\mathrm{symmetric}}=\{(1,7);(7,1);(7,4);(4,7);(4,4)\}$\\
The symmetric closure for R3 is $\mathrm{R}3_{\mathrm{symmetric}}=\{(1,1);(1,7);(4,4);(7,1);(7,4);(4,7);(7,7);(8,8)\}$\\
iv. The \{reflexive, symmetric, transitive\}\--closure is the equivalence closure. Hence,\\
the equivalence closure for R1 is $\mathrm{R}1_{\mathrm{equiv}} = \{(1,7);(1,8);(7,4);(4,7);(1,1);(4,4);(7,7);(8,8);(1,4);(7,1);(8,1)\}$\\
the equivalence closure for R3 is $\mathrm{R}3_{\mathrm{equiv}}=\{(1,1);(1,7);(4,4);(7,1);(7,4);(4,7);(7,7);(8,8);(1,4);(4,1)\}$}
\pagebreak

% Cau 2

\qs{Answer the following question}{
Consider the set $S=\{\mathrm{a}, \mathrm{b}, \mathrm{c}, \mathrm{d}, \mathrm{e}, \mathrm{f}\}$ and the relations:
\newline
$\mathrm{f}1=\{(\mathrm{a}, \mathrm{a}),(\mathrm{a}, \mathrm{~b}),(\mathrm{c}, \mathrm{~d}),(\mathrm{e}, \mathrm{f})\} \\
\mathrm{f}2=\{(\mathrm{a}, \mathrm{~b}),(\mathrm{b}, \mathrm{c}),(\mathrm{c}, \mathrm{~d}),(\mathrm{e}, \mathrm{~d})\} \\
\mathrm{f}3=\{(\mathrm{a}, \mathrm{a}),(\mathrm{b}, \mathrm{~b}),(\mathrm{c}, \mathrm{c}),(\mathrm{d}, \mathrm{~d}),(\mathrm{e}, \mathrm{e}),(\mathrm{f}, \mathrm{f})\} \\
\mathrm{f}4=\{(\mathrm{a}, \mathrm{f}),(\mathrm{b}, \mathrm{~b}),(\mathrm{c}, \mathrm{~d}),(\mathrm{e}, \mathrm{f})\}$\\ 
Complete the following table (by writing yes or no):\\
    \begin{tabular}{|l|l|l|l|l|}
    \hline & Function & 1-to-1 & Onto & 1-to-1 correspondence \\
    \hline f1 & & & & \\
    \hline f2 & & & & \\
    \hline f3 & & & & \\
    \hline f4 & & & & \\
    \hline
    \end{tabular}
}

\sol{\newline
\begin{tabular}{|l|l|l|l|l|}
    \hline & Function & 1-to-1 & Onto & 1-to-1 correspondence \\
    \hline f1 &no &no &no &no \\
    \hline f2 &yes &no &no &no \\
    \hline f3 &yes &yes &yes &yes \\
    \hline f4 &yes &no &no &no \\
    \hline
    \end{tabular}
}\\
\newline

% Cau 3

\qs{Prove by induction on $n$ that:}{
$\sum_{i=0}^n i^3=\left(\sum_{i=0}^n i\right)^2$
}

\sol{\newline
$\bullet$ Base case: $n=0 \implies P(0) \equiv 0^{3} = 0^{2}$ (true).\\ 
$\bullet$ Since the base case is true, we assume the inductive hypothesis $P(k)$ is true for some $k$. Hence, $P(k) \equiv$ for some $k$, the following equality holds:  $\sum_{i=0}^k i^3=\left(\sum_{i=0}^k i\right)^2$.
 Now, we prove that $P(k+1)$ is also true.\\
Observe that, $P(k+1) \equiv \sum_{i=0}^{k+1} i^3=\left(\sum_{i=0}^{k+1} i\right)^2$ (*)\\
$\Leftrightarrow \mathrm{LHS} = \sum_{i=0}^{k} i^{3} + (k+1)^{3} = \left(\sum_{i=0}^k i\right)^2 + (k+1)^{3}$\\
Hence, plug LHS back into (*), we obtain: 
\begin{align*}
 \left(\sum_{i=0}^k i\right)^2 + (k+1)^{3} &= \left(\sum_{i=0}^{k+1} i\right)^2\\
\Leftrightarrow (k+1)^{3}                  &= \left(\sum_{i=0}^{k+1} i\right)^2 - \left(\sum_{i=0}^k i\right)^2\\
\Leftrightarrow (k+1)^{3}                  &= \biggr[\left(\sum_{i=0}^{k+1} i\right) - \left(\sum_{i=0}^k i\right)\biggr] . \biggr[\left(\sum_{i=0}^{k+1} i\right) + \left(\sum_{i=0}^k i\right)\biggr] \\
\Leftrightarrow (k+1)^{3}                  &= \biggr[\left(\sum_{i=0}^{k} i\right) + (k+1) - \left(\sum_{i=0}^k i\right)\biggr] . \biggr[\left(\sum_{i=0}^{k} i\right) + (k+1) + \left(\sum_{i=0}^k i\right)\biggr]\\
\Leftrightarrow (k+1)^{3}                  &= \left(k+1\right) . \left(\frac{k(k+1)}{2} + (k+1) + \frac{k(k+1)}{2}\right)\\
\Leftrightarrow (k+1)^{3}                  &= \left(k+1\right)\left((k+1) + k(k+1)\right) = \left(k+1\right) \left(k+1\right)\left(k+1\right) \blacksquare\\
\end{align*}
Hence, $P(k+1)$ is true, proving $P(n)$ true $\forall n$ by mathematical induction.}

% Cau 4

\qs{Answer the following question}{
Prove that for a finite set $A,\left|2^{\mathrm{A}}\right|=2^{|\mathrm{A}|}$
}

\sol{
Say, the finite set A has the number of elements is $n$. Hence, the statement to be proven becomes: Given a set A, $|A|=n$; prove that $|2^{A}| = 2^{n}$.\\
We know that $2^{A}$ indicates the power set of the set A. So, $|2^{A}|$ indicates the size of the power set of the set A, or $|\mathscr{P}(A)|$.\\
\newline
$\bullet$ Base case: $n=1$, say $A = \{a\}$. $|\mathscr{P}(A)| = 2^1 = 2 = \{a, \emptyset\}$ (true).\\
$\bullet$ Since the base case is true, we assume the inductive hypothesis $P(k)$ is true for some $k$. Hence, we show that $P(k+1)$ is true. Or, given a set A, $|A|=k+1$; prove that $|\mathscr{P}(A)| = 2^{k+1}$.\\
\newline
Since $P(k)$ is assumed to be true, $|A| = k$ and $|\mathscr{P}(A)| = 2^{k}$. Let the addition set of 1 element making $|A| = k+1$ is $T=\{z\}$. Note that, T and A are assumed to be disjoint.\\
Now, the power set of $\left(A \cup T\right)$ contains 2 components. The set of subsets of $T$ and the union of the set of subsets of $T$ and the element (or subsets) of the original set $A$.\\
$\implies |\mathscr{P}(A \cup T)| = 2^{k}.2^{k} = 2^{k+1}~\blacksquare$.\\
\newline
Hence, $P(k+1)$ is also true. Proving $P(n)$ true $\forall n$ by mathematical induction.}\\
\newline

% Cau 5

\qs{Answer the following question}{
Show that if $S_{1}$ and $S_{2}$ are finite sets with $|S_{1}| = n$ and $|S_{2}| = m$, then $|S_{1} \cup S_{2}| \leq n + m$.}

\sol{\newline
According to inclusion-exclusion principle for two sets $S_{1}$ and $S_{2}$:
\begin{align*}
    |S_{1} \cup S_{2}| &= |S_{1}| + |S_{2}| - |S_{1} \cap S_{2}|\\
    |S_{1} \cup S_{2}| &= n + m - |S_{1} \cap S_{2}|\\
\end{align*}
We know that, $|S_{1} \cap S_{2}| \geq 0$ (= 0 when $S_{1}$ and $S_{2}$ are disjoint sets.) Then, $-|S_{1} \cap S_{2}| \leq 0$. Hence, 
\begin{align*}
    |S_{1} \cup S_{2}| &= n + m - |S_{1} \cap S_{2}| \leq n + m~ \blacksquare
\end{align*}
Hence, $|S_{1} \cup S_{2}| \leq n + m ~\forall n, m$.
}

\end{document}
