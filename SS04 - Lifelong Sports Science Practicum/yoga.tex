\documentclass[a4paper,10.5pt]{article}
\usepackage[utf8]{inputenc}
\usepackage{amsmath}
\usepackage{geometry}
\usepackage{setspace}
\geometry{a4paper, margin=1in}
\setstretch{1.5}

\title{SS04 Health and Sports Sciences Practicum\\Yoga: Sun Salutation Practice Record}

\author{NGUYEN Tuan Dung - s1312004}
\date{November 19, 2024}

\begin{document}

\maketitle

\section*{1. Implementation Date and Start Time}
I started reading and doing the exercises on Sunday morning, November $3^{\mathrm{rd}}$, at approximately 11:00 AM, when I woke up.
\section*{2. Predictions Before Implementation}
Before starting the Sun Salutation sequence, I honestly thought I’d notice some big changes in both my mind and body after doing it ten times.

Physically, I was hoping to get more flexible, especially in my hamstrings, shoulders, and back, since poses like Uttānāsana, Adho Mukha Svanāsana, and Bhujangāsana are all about stretching those areas. I also figured I’d get stronger in my arms, core, and legs, especially with all the Plank Poses and lunges. I even thought my posture might improve a bit because Tadasana and the flow, in general, really focus on alignment.

Mentally, I was looking forward to feeling calmer and more balanced. The way the sequence connects movement with breathing seemed like it could help me de-stress and focus better. I also imagined it would give me a little energy boost since yoga is great for circulation and releasing tension.

\section*{3. Post-Implementation Review}
After completing the Sun Salutation sequence ten times, I realized the physical and mental changes that occurred. Some results were as predicted.

I noticed a big improvement in my flexibility after practicing. At first, my hamstrings and shoulders felt pretty tight, but by the $10^{\mathrm{th}}$ round, I was able to move between the poses much more smoothly and comfortably. For example, my Adho Mukha Svanasana felt a lot more natural, and I could fold deeper in Uttanasana without that nagging tension.

I also felt myself getting stronger. Plank Pose, which was really tough at first, became easier to hold by the end, with less strain on my core and arms. My legs felt noticeably stronger too, especially during the lunges. What I didn’t see coming was the soreness in my wrists and thighs during the first few sessions. That taught me to pay closer attention to how I was engaging my muscles and to spread out my weight more evenly, especially in weight-bearing poses like Plank and Ashwa Sanchalanasana.

Mentally, this practice helped me feel more grounded and relaxed. The breathing part made a huge difference—it really helped me calm down and stay focused. By the $10^{\mathrm{th}}$ round, I found myself letting go of random distractions and just being fully in the moment with the flow.

What surprised me the most, though, was how connected I felt to my body. Doing the sequence over and over made me notice little things, like how my spine aligned or how my feet felt planted on the mat. That awareness gave me a real confidence boost and a sense of accomplishment that I didn’t expect.

\section*{4. Comfortable and Challenging Poses}
\subsection*{Comfortable Poses}
\begin{itemize}
    \item \textbf{Tadasana:} This pose was simple and grounding. It felt like a moment of calm and reset at the beginning and end of the sequence. Standing tall and focusing on my breath helped me prepare for the more dynamic movements.
    \item \textbf{Bhujangasana:} I enjoyed this pose because it offered a gentle stretch for my back and opened up my chest. It felt relaxing, and the transition from Plank to Cobra gave me a brief moment of rest.
    \item \textbf{Adho Mukha Svanasana:} Although it took some effort to hold this pose initially, it became more comfortable as my shoulders and hamstrings loosened. I found it refreshing as it gave me a good stretch and allowed me to pause for a moment during the flow.
\end{itemize}

\subsection*{Challenging Poses}
\begin{itemize}
    \item \textbf{Plank:} This pose was one of the hardest for me. Holding it for even a few seconds was tough, especially during the first few rounds. My core and arms felt fatigued quickly, but with practice, I noticed small improvements in my stability and strength.
    \item \textbf{Ashwa Sanchalanasana:} Balancing in this pose was challenging, especially when transitioning between sides. My legs wobbled, and I had to focus intently on keeping my core engaged to maintain stability.
    \item \textbf{Uttanasana:} This pose was unexpectedly difficult at first due to tight hamstrings. I struggled to fold deeply without feeling strain in my lower back. However, as I practiced more, I noticed my hamstrings gradually loosening, which made the pose more manageable by the $10^{\mathrm{th}}$ round.
\end{itemize}

\end{document}
